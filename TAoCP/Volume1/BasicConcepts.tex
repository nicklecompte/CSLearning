\documentclass{article}

\usepackage{amsmath}

\begin{document}

\section{Algorithms}

A sketch of a formal definition of an algorithm:

A \emph{computational method} is a quadruple $(Q,I,\Omega,f)$, where
\begin{itemize}
  \item $I\subset Q$ is the \emph{input}
  \item $\Omega \subset Q$ is the \emph{output}
  \item $f : Q \rightarrow Q$ is the \emph{computational rule}, which satisfies $f(\omega) = \omega\ \forall \omega \in \Omega$.
\end{itemize}

Each $x\in I$ defines a \emph{computational sequence} $x_0,x_1,\dots,$, where $x_0=x$ and $x_{k+1} = f(x_k)$. The sequence \emph{terminates} in $k$ steps if $k$ is the smallest integer such that $x_k \in \Omega$. An \emph{algorithm} is a computational method that terminates in some finite number of steps for all $x$ in $I$.

As an example, we present Euclid's algorithm in this formalization: let $Q$ be the set of all singletons $\{n\}$, all ordered pairs $(m,n)$, and all ordered quadruples $(m,n,r,1),(m,n,r,2),(m,n,p,3)$ where $m,n,p$ are positive integers and $r$ is a nonnegative integer. Let $I$ be the ordered pairs $(m,n)$ and $\Omega$ the set of singletons $\{n\}$. Define $f$ by

\begin{align}
  f((m,n)) = (m,n,0,1);\ f((n)) = (n); \\
  f((m,n,r,1)) = (m,n, m\ \% \ n, 2); \\
  f((m,n,r,2)) = (n)\ \mathrm{if}\ r=0,\ (m,n,r,3)\ \mathrm{otherwise}; \\
  f((m,n,p,3)) = (n,p,p,1)
\end{align}


\section{Algorithms - solutions to exercises}
\
\textbf{1.1} $t \leftarrow a$, $a \leftarrow b$, $b \leftarrow c$, $c \leftarrow d$, $d\leftarrow t$.

\

\textbf{1.2} We have $m \leftarrow n$ and $n \leftarrow r$. Since $r < n$, after assignment $n < m$.

\

\textbf{1.3} \textbf{Algorithm F}. Given two positive integers $m$ and $n$, find the greatest common divisor.

\ \textbf{F1} Divide $m$ by $n$.

\ \textbf{F2} Set $m$ equal to the remainder.

\ \textbf{F3} If $m=0$ then the answer is $n$.

\ \textbf{F4} Otherwise divide $n$ by $m$.

\ \textbf{F5} Set $n$ equal to the remainder.

\ \textbf{F6} If $n=0$ then the answer is $m$.

\ \textbf{F7} Go to \textbf{F1}.

\
\textbf{1.4} $6099 \% 2166 = 1767 \Rightarrow 2166 \% 1767 = 399$

$\Rightarrow 1767 \%  399 = 171 \Rightarrow 399 \$ 171 = 57 \Rightarrow 171 \% 57 = 0$.

So the GCD is 57.
\

\textbf{1.5} Not finite, not definite, not effective.

\textbf{1.6}
$n=1$: $1 \% 5 = 1 \Rightarrow 5 \% 1 = 0$, 2 steps 

$n=2$: $2 \% 5 = 2 \Rightarrow 5 \% 2 = 1 \Rightarrow 2 \% 1 = 0$, 3 steps

$n=3$: $3 \% 5 = 3 \Rightarrow 5 \% 3 = 2 \Rightarrow 3 \% 2 = 1 \Rightarrow 2 \%1 = 0$, 4 steps

$n=4$: $4 \% 5 = 4 \Rightarrow 5 \% 4 = 4 \Rightarrow 4 \% 4 = 0$, 3 steps 

$n=5$: $5 \% 5 = 0$, 1 step

So $T_5 = 2.6$.

\
\textbf{1.7} $U_m$ is well-defined: if $n > m$, the first step of the Euclidean algorithm simply swaps $n$ and $m$ (since $m\ \%\ n = m$) and $U_m = T_m + 1$. If $n < m$ then there are only finitely many cases.


\section{Mathematical Preliminaries - Induction}

Algorithmic proof procedure:

\textbf{Algorithm I - Construct a proof} Given a positive integer $n$ and proposition $P(n)$, this algorithm will output a proof that $P(n)$ is true (if it succeeds).

\textbf{I1} [Prove $P(1)$] Set $k \leftarrow 1$ and use another algorithm to output a proof of $P(1)$.

\textbf{I2} [$k=n$?] If $k=n$, terminate - the required proof was found in the previous step.

\textbf{I3} [$k<n$] Otherwise $k<n$. Use another algorithm to output a proof of the following statement: ``If $P(1)$, $P(2)$,$\dots$,$P(k)$ is true, then $P(k+1)$ is true.'' Then output the statement ``We have already proved $P(1),\dots,P(k)$, hence $P(k+1)$ is true.'' Combine these statements.

\textbf{I4} Set $k \leftarrow k + 1$. Go to step \textbf{I2}.

Here is an inductive proof of a fact about the Fibonacci sequence. Let $F_0 = 0,F_1=1$ and $F_n = F_{n-1} + F_{n-2}$ for $n\geq 2$. Define $\phi = (1 + \sqrt{5})/2$. Then $F_n \leq \phi^{n-1}$ for all positive $n$.

We proceed according to the algorithm above. This is clearly true for $n=0$ and $n=1$,l so we have obtained a proof of $P(1)$. For $P(2)$, $F_2 = 1$ and $\phi > 1.6$, so we have a (computational) proof of $P(2)$. Now assume our target is $k+1$ with $k>1$ and we have $k$ proofs $P(1),\dots,P(k)$. Since $F_{k+1} = F_k + F_{k-1}$, and by hypothesis $F_k \leq \phi^{k-1}$ and $F_{k-1} \leq \phi^{k-2}$,

\begin{equation}
  F_{k+1} \leq \phi^{k-1} + \phi^{k-2} = \phi^{k-2}(1+\phi).
\end{equation}

$\phi$ is actually the positive solution to $1 + \phi = \phi^2$. So plugging this in gives $F_{k+1} \leq \phi^k$, as desired.

Note that our proof would have failed if we didn't have direct proofs of $P(1)$ and $P(2)$: $P(1)$ would fail at the inductive step since the theorem is not true for $n=0$, and for $P(2)$ we couldn't have applied the method at the $F_{k-1} \leq \phi^{k-2}$ step (since $k=1$).

\end{document}
